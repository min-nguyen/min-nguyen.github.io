\documentclass[acmsmall, 9pt]{article}

\usepackage[a4paper, total={6in, 10in}]{geometry}
\newcommand\hmmax{0} % http://tex.stackexchange.com/questions/3676
\newcommand\bmmax{0}

\usepackage{float}
\usepackage{mathpartir}
\usepackage{mathtools}
\usepackage{amssymb}
\usepackage{stmaryrd}
\usepackage{bm}
\usepackage{listings}
\usepackage{enumitem}
\usepackage{subcaption} 
\usepackage{xcolor}
\usepackage[toc,page]{appendix}
\usepackage{titlesec}
\usepackage{amsmath}
% \usepackage{tikzit}
% \input{sample.tikzstyles}
\usepackage[font=small, labelfont=bf, textfont=normalfont]{caption}
\usepackage[backend=bibtex]{biblatex}
\usepackage{hyperref}
\hypersetup{
    colorlinks=true,
    linkcolor=blue,
    filecolor=magenta,
    urlcolor=cyan
    } 
\definecolor{burgundy}{rgb}{0.5, 0.0, 0.13}
\definecolor{lava}{rgb}{0.81, 0.06, 0.13}
\definecolor{bazaar}{rgb}{0.6, 0.47, 0.48}
\definecolor{brown(web)}{rgb}{0.65, 0.16, 0.16}
\definecolor{lavenderindigo}{rgb}{0.58, 0.34, 0.92}
% \definecolor{davy\'sgrey}{rgb}{0.33, 0.33, 0.33}
\def\sectionautorefname{\textsection}
\def\subsectionautorefname{\textsection}
\def\subsubsectionautorefname{\textsection}
% \titleformat*{\section}{\large\bfseries}
% \titleformat*{\subsection}{\normalsize\bfseries}
% \titleformat*{\subsubsection}{\itshape}
% \titlespacing*{\subsubsection}{0em}{2ex}{0.5em}
\newcommand{\secref}[1]{{\color{burgundy}{\autoref{#1}}}}
\newcommand{\todo}[1]{{\color{bazaar}{To do: #1}}}
\newcommand{\lineref}[1]{{\color{blue} \ref{#1}}}
\newcommand{\figref}[1]{{\color{lavenderindigo}{\ref{#1}}}}
\newcommand{\Epsilon}{\mathcal{E}}


% Math
\newcommand*{\tyFun}[2]{{#1}\rightarrow{#2}}
\newcommand*{\lbreak}{\vspace{0.3cm}\noindent}
\newcommand*{\concat}{\cdot}

% Math Figures.
\newenvironment{mathfig}{\begin{sdisplaymath}}{\end{sdisplaymath}}
\newenvironment{syntaxfig}{\begin{mathfig}\begin{array}{@{}l@{\quad}r@{~~}c@{\quad}ll}}{\end{array}\end{mathfig}}
\newenvironment{nop}{}{}
\newenvironment{smathpar}
   {\begin{nop}\small\begin{mathpar}}
   {\end{mathpar}\end{nop}\ignorespacesafterend}

\definecolor{highlightcolor}{rgb}{1.0,0.8,0.8}
\definecolor{shadecolor}{rgb}{0.9,0.9,0.9}
\definecolor{lightgray}{rgb}{0.8,0.8,0.8}

\newcommand*{\shadebox}[1]{\fcolorbox{lightgray}{shadecolor}{\raisebox{0pt}[0.60\baselineskip][0.05\baselineskip]{#1}}}
\newcommand*{\ruleName}[1]{\textnormal{\textsf{#1}}}
\newcommand*{\kw}[1]{\texttt{#1}}
\newcommand*{\exLet}[3]{\kw{let}\;{#1}\;=\;{#2}\;\kw{in}\;{#3}}
\lstset{
  frame=none,
  xleftmargin=2pt,
  stepnumber=1,
  numbersep=5pt,
  numberstyle=\ttfamily\tiny\color[gray]{0.3},
  belowcaptionskip=\bigskipamount,
  captionpos=b,
  escapeinside={*'}{'*},
  language=haskell,
  tabsize=2,
  emphstyle={\bf},
  commentstyle=\it,
  stringstyle=\mdseries\rmfamily,
  showspaces=false,
  keywordstyle=\bfseries\rmfamily,
  columns=flexible,
  basicstyle=\small\sffamily,
  showstringspaces=false,
  morecomment=[l]\%,
  % morekeywords={fold},
  deletekeywords={words, String, length, zip, replicate, map, insert, Nothing, Just, fromMaybe, head, fst, mapM, Int, Double, Bool},
  escapeinside={(*}{*)},
  escapechar=|
}

\setlist[itemize]{leftmargin=4mm}
\setlist[enumerate]{leftmargin=4mm}
\addbibresource{../bibliography.bib}
\begin{document}
\pagestyle{empty}

\title{Lambda Calculus -- $\lambda^{\rightarrow}$, System F, and System F$_\omega$ }

\maketitle

\section{Simply Typed Lambda Calculus ($\lambda^{\rightarrow}$)}
Simply typed lambda calculus \cite{cambridge-lambda-calc} is also traditionally called $\lambda^{\rightarrow}$, where the arrow $\rightarrow$ indicates the centrality of function types $\tyFun{A}{B}$. The elements of lambda calculus are divided into three ``sorts'':
\begin{itemize}
  \item \textbf{terms} ranged over by metavariables $M, N$.
  \item \textbf{types} ranged over by metavariables $A, B$. We write $M: A$ to say type $M$ has type $A$.
  \item \textbf{kinds} ranged over by metavariable $K$. We write $T : K$ to say type $T$ has kind $K$.
\end{itemize}
\noindent
The grammar of $\lambda^{\rightarrow}$ is given by:
\begin{align*}
  \text{Kinds} \quad K &::= * \\
  \text{Types} \quad  A, B &::= \iota \; | \;  A \rightarrow B  \\
  \text{Raw terms} \quad M, N &::= c \; | \; x \; | \;  \lambda x^A . \, M \; | \; M \, N
\end{align*}

\noindent
\textbf{Kinds} Kinds play little part in $\lambda^{\rightarrow}$, so their structure trivially consists just of $*$ i.e. the kind of value types.

\noindent
\textbf{Types} Types consist of base types $\iota$ such as integers and booleans, and functions where $A \rightarrow B$ represents a function taking a type $A$ to a type $B$.

\noindent
\textbf{Terms} Term variables are written $x$. Constants are represented by terms $c$. The term $\lambda x^A . \, M$ (also written $\lambda x : A . \, M$) is a function which when given some term of type $A$, binds it to the variable $x$ and returns the term $M$. Lastly we have application $M \; N$ which applies a term $M$ to a term $N$.

\begin{figure}[H]
\flushleft \shadebox{$\Delta \vdash A : K$}
\begin{smathpar}
  \inferrule*[lab={\ruleName{constant}}]
  {
     \strut
  }
  {
     \Delta \vdash \iota : *
  }
  \and
  \inferrule*[lab={\ruleName{function}}]
  {
    \Delta \vdash A : *
    \\
    \Delta \vdash B : *
  }
  {
     \Delta \vdash \tyFun{A}{B} : *
  }
\end{smathpar}
\caption{Kinding Rules ($\lambda^{\rightarrow}$)}
\end{figure}

\begin{figure}[H]
\flushleft \shadebox{$\Gamma \vdash M : A$}
\begin{smathpar}
  \inferrule*[lab={\ruleName{constant}}]
  {
     \strut
  }
  {
     \Gamma \vdash c : \iota
  }
  \and
  \inferrule*[lab={\ruleName{var}}]
  {
      x : A \in \Gamma
  }
  {
     \Gamma \vdash x : A
  }
  \and
  \inferrule*[lab={\ruleName{lambda}}]
  {
     \Gamma \concat (x : A) \vdash M : B
  }
  {
     \Gamma \vdash \lambda x^A. \, M :  \tyFun{A}{B}
  }
  \and
  \inferrule*[lab={\ruleName{application}}]
  {
     \Gamma \vdash M : \tyFun{A}{B}
     \\
     \Gamma \vdash N : A
  }
  {
     \Gamma \vdash M \, N : B
  }
\end{smathpar}
\caption{Typing Rules ($\lambda^{\rightarrow}$)}
\end{figure}

\section{Polymorphic Typed Lambda Calculus (System F)}
System F \cite{lambda-calc, cambridge-lambda-calc}, also known as polymorphic lambda calculus or second-order lambda calculus, is a typed lambda calculus that extends simply-typed lambda calculus. It extends this by adding support for ``type-to-term'' abstraction, allowing polymorphism through the introduction of a mechanism of universal quantification over types. It therefore formalizes the notion of parametric polymorphism in programming languages. It is known as second-order lambda calculus because from a logical perspective, it can describe all functions that are provably total in second-order logic.


\lbreak
The grammar of System F is given by:
\begin{align*}
  \text{Kinds} \quad K &::= * \\
  \text{Types} \quad  A, B &::= \iota \; | \; A \rightarrow B \; | \; \alpha \; | \; \forall \alpha^K. \, A  \\
  \text{Terms} \quad M, N &::= x \; | \; \lambda x^A . \, M \; | \; M \, N  \; | \; \Lambda \alpha^K . \, M \; | \; M \, [A]
\end{align*}

\noindent
\textbf{Kinds} Kinds remain the same, and all types have kind $*$.

\noindent
\textbf{Types} We extend types $A, B$ with (polymorphic) type variables $\alpha$ and universally quantified types $\forall \alpha^{\kappa}. \, A$ in which the bound type variable $\alpha$ of kind $K$ may appear in $A$ (we note that the only kind $K$ in System F is $*$). An important point to note is that type variables $\alpha$ are only well-formed if they exist within the scope of which they are quantified by $\forall \alpha$. We note that in a polymorphic lambda calculus without a type scheme, such as this one, it is possible for type variables $\alpha$ to appear on their own without being bound to an inscope quantifier $\forall \alpha$ -- therefore this grammar on its own does not ensure well-formed types.

\noindent
\textbf{Terms} Lambda abstraction $\lambda x^A . \, M$ can now take variables $x$ which have universally quantified types, e.g. $\forall \alpha. \, \alpha$. We extend terms with type abstraction $\Lambda \alpha^K . \, M$ (also written $\Lambda \alpha : K . \, M$) whose parameter $\alpha$ is a type of kind $K$ and returns a term $M$. We can then apply types $A$ to type lambda abstractions $M$ using type application $M \, [A]$.

\begin{figure}[H]
\flushleft \shadebox{$\Delta \vdash T : K$}
\begin{smathpar}
  \inferrule*[lab={\ruleName{constant}}]
  {
     \strut
  }
  {
     \Delta \vdash \iota : *
  }
  \and
  \inferrule*[lab={\ruleName{function}}]
  {
     \Delta \vdash A : *
     \\
     \Delta \vdash B : *
  }
  {
     \Delta \vdash \tyFun{A}{B} : *
  }
  \and
  \inferrule*[lab={\ruleName{forall}}]
  {
     \Delta \concat (\alpha : K) \vdash A : *
  }
  {
     \Delta \vdash \forall \alpha^K. \, A : *
  }
  \and
  \inferrule*[lab={\ruleName{type variable}}]
  {
     \alpha : K \in \Delta
  }
  {
     \Delta \vdash \alpha : K
  }
\end{smathpar}
\caption{Kinding Rules (System F)}
\end{figure}

\begin{figure}[H]
\flushleft \shadebox{$\Gamma \vdash M : A$}
\begin{smathpar}
  \inferrule*[lab={\ruleName{var}}]
  {
     x : A \in \Gamma
  }
  {
     \Gamma \vdash x : A
  }
  \and
  \inferrule*[lab={\ruleName{lambda abstraction}}]
  {
     \Gamma \concat (x : A) \vdash M : B
  }
  {
     \Gamma \vdash \lambda x^A. \, M : \tyFun{A}{B}
  }
  \and
  \inferrule*[lab={\ruleName{application}}]
  {
     \Gamma \vdash M : \tyFun{A}{B}
     \\
     \Gamma \vdash N : A
  }
  {
     \Gamma \vdash M \, N : B
  }
  \and
  \inferrule*[lab={\ruleName{type abstraction}}]
  {
     \Delta \concat (\alpha : K) \vdash M : A
  }
  {
     \Gamma \vdash \Lambda \alpha^K. \, M : \forall \alpha^K. \, A
  }
  \and
  \inferrule*[lab={\ruleName{type application}}]
  {
     \Gamma \vdash M : \forall \alpha^K. A
     \\
     \Delta \vdash B : K
  }
  {
     \Gamma \vdash M \, [B] : A[\alpha \mapsto B]
  }
\end{smathpar}
\caption{Typing Rules (System F)}
\end{figure}


\section{Higher-Order Polymorphic Typed Lambda Calculus (System F$_\omega$)}
System F$_\omega$ \cite{cambridge-lambda-calc, pierce2002types}, also known as higher-order polymorphic lambda calculus, extends System F with richer kinds and adds type-level lambda-abstraction and application.


\subsubsection{System F$_\omega$}
\begin{align*}
  \text{Kinds} \quad K &::= * \; | \; K_1 \rightarrow K_2\\
  \text{Types} \quad  A, B &::= \iota \; | \;  A \rightarrow B \; | \; \forall \alpha^K . \, A\; | \; \alpha \; | \; \lambda \alpha^K. \, A \; | \; A \, B\\
  \text{Terms} \quad M, N &::= x \; | \; \lambda x^A . \, M \; | \; M \, N  \; | \; \Lambda \alpha^K . \, M \; | \; M \, [A]
\end{align*}

\noindent
\textbf{Kinds} In System F, the structure of kinds has been trivial, limited to a single kind $*$ to which all type expressions belonged. In System F$_\omega$, we enrich the set of kinds with an operator $\rightarrow$ such that if $K_1$ and $K_2$ are kinds, then $K_1 \rightarrow K_2$ is a kind. This allows us to construct kinds which contain \textit{type operators/constructors} and higher-order forms of these, such as product $\times$. We are then free to extend this calculus with arbitrary custom kind constants.

\noindent
\textbf{Types} The set of types in System F$_\omega$ additionally includes type constructors i.e. type-level lambda-abstraction $\lambda \alpha^K. \, A$, which when provided a type of kind $K$, binds this to the type variable $\alpha$ and returns the type $A$. Type constructors $A$ can be applied to a type $B$ to form a new type $A\, B$. Universal quantification $\forall \alpha^K . \, A$ now requires the bound type variable $\alpha$ to be annotated by a kind $K$, meaning types can be parameterised by polymorphic type variables of any kind $K$.

%($A\,B : K_2$ when $A : K_1 \rightarrow K_2$ and $B : K_1$) as we are able to apply higher-kinded types $K_1 \rightarrow K_1$ to other types.


\noindent
\textbf{Terms} Although the terms in System F$_\omega$ remain the same as System F, the term for type abstraction ($\Lambda \alpha^K . \, M$) can now take types with kinds other than $*$.

\lbreak
The introduction of richer kinds means that it becomes more necessary to add \textit{kinding rules} to dictate what are well-formed types.

\begin{figure}[H]
\flushleft \shadebox{$\Delta \vdash T : K$}
\begin{smathpar}
  \inferrule*[lab={\ruleName{constant}}]
  {
     \strut
  }
  {
     \Delta \vdash \iota : *
  }
  \and
  \inferrule*[lab={\ruleName{function}}]
  {
     \Delta \vdash A : *
     \\
     \Delta \vdash B : *
  }
  {
     \Delta \vdash \tyFun{A}{B} : *
  }
  \and
  \inferrule*[lab={\ruleName{forall}}]
  {
     \Delta \concat (\alpha : K) \vdash A : *
  }
  {
     \Delta \vdash \forall \alpha^K. \, A : *
  }
  \and
  \inferrule*[lab={\ruleName{type variable}}]
  {
     \alpha : K \in \Delta
  }
  {
     \Delta \vdash \alpha : K
  }
  \and
  \inferrule*[lab={\ruleName{type constructor}}]
  {
     \Delta \concat (\alpha : K_1) \vdash A : K_2
  }
  {
     \Delta \vdash \lambda \alpha^{K_1} . \, A : \tyFun{K_1}{K_2}
  }
  \and
  \inferrule*[lab={\ruleName{type constructor application}}]
  {
     \Delta \vdash A : \tyFun{K_1}{K_2}
     \\
     \Delta \vdash B : K_1
  }
  {
     \Delta \vdash A \, B : K_2
  }
\end{smathpar}
\caption{Kinding Rules (System F$_\omega$)}
\end{figure}

\begin{figure}[H]
\flushleft \shadebox{$\Gamma \vdash M : A$}
\begin{smathpar}
  \inferrule*[lab={\ruleName{var}}]
  {
     x : A \in \Gamma
  }
  {
     \Gamma \vdash x : A
  }
  \and
  \inferrule*[lab={\ruleName{lambda abstraction}}]
  {
     \Gamma \concat (x : A) \vdash M : B
  }
  {
     \Gamma \vdash \lambda x^A. \, M : \tyFun{A}{B}
  }
  \and
  \inferrule*[lab={\ruleName{application}}]
  {
     \Gamma \vdash M : \tyFun{A}{B}
     \\
     \Gamma \vdash N : A
  }
  {
     \Gamma \vdash M \, N : B
  }
  \and
  \inferrule*[lab={\ruleName{type abstraction}}]
  {
     \Delta \concat (\alpha : K) \vdash M : A
  }
  {
     \Gamma \vdash \Lambda \alpha^K. \, M : \forall \alpha^K. \, A
  }
  \and
  \inferrule*[lab={\ruleName{type application}}]
  {
     \Gamma \vdash M : \forall \alpha^K. A
     \\
     \Delta \vdash B : K
  }
  {
     \Gamma \vdash M \, [B] : A[\alpha \mapsto B]
  }
\end{smathpar}
\caption{Typing Rules (System F$_\omega$)}
\end{figure}

\printbibliography
\end{document}